\documentclass[12pt,a4paper]{article}

% ------------------ Basic setup ------------------
\usepackage[utf8]{inputenc}
\usepackage[T1]{fontenc}
\usepackage{graphicx}
\usepackage{geometry}
\geometry{margin=1in}

% Header & footer
\usepackage{fancyhdr}
\pagestyle{fancy}

\usepackage{lastpage}    

\newcommand{\studentname}{Zhidong Zhang}
\newcommand{\course}{Neural Modeling}
\newcommand{\assignmentnumber}{Homework 4}
\newcommand{\assignmentdate}{\today}
\renewcommand{\figurename}{Fig.}

% ------------------ Header configuration ------------------
\lhead{\studentname}
\chead{\course\ -\ \assignmentnumber}
\rhead{\assignmentdate}
% \cfoot{\thepage}
\cfoot{\thepage\ / \pageref{LastPage}}  % i/n的页数形式
\rfoot{}

% ------------------ Typography ------------------
\usepackage{setspace}
\setstretch{1.3}  % line spacing

% ------------------ Math packages ------------------
\usepackage{amsmath, amssymb}

\renewcommand{\thesubsection}{\Alph{subsection}}

% ------------------ Document ------------------
\begin{document}

% \section{Online Course}
\section{Online course}

The summary of the online course is attached to the end of this report.

\section{Exercises}

\subsection{Cone Sensitivity}

To fascilitate the following calculations, interpolation is performed on the cone sensitivity data to obtain values per 0.5 nm. It is done by using the \texttt{interp1d} function from the \texttt{scipy.interpolate} module in Python with the \texttt{cubic} method, to make smooth curves.

The cone sensitivity $f_{a}(\lambda)$ for each type of cone $a \in \{L, M, S\}$ is shown in Fig.~\ref{fig:cone_sensitivity}.

\begin{figure}[h]
    \centering
    \includegraphics[width=0.7\textwidth]{figures/Cone_Sensitivity_Spectrum.pdf}
    \caption{Cone sensitivity spectrum $f_a(\lambda)$ for each type of cone $a \in \{L, M, S\}$.}
    \label{fig:cone_sensitivity} 
\end{figure}


\subsection{Likelihoods on a given input}

Given a input $\mathbf{S}(\lambda, I)$ with $I=100$ and $\lambda = 570$ nm, the likelihood $P(r_a|\mathbf{S}(\lambda, I))$ and the mean cone absorption rate $\overline{r} = If_a(\lambda)$ for each type of cone $a \in \{L, M, S\}$ is shown in Fig.~\ref{fig:likelihood_570nm_100I}.
\begin{figure}[h]
    \centering
    \includegraphics[width=0.7\textwidth]{figures/Poisson_Distribution_At_570nm_100I.pdf}
    \caption{Likelihood $P(r_a|\mathbf{S}(\lambda, I))$ at $\lambda = 570$ nm and $I=100$ for each type of cone $a \in \{L, M, S\}$. The dashed lines indicate the average absorption rates $\overline{r}$.}
    \label{fig:likelihood_570nm_100I} 
\end{figure}

\subsection{Samples and 3D Visualization}

Here 500 samples are drawn from the Poisson distributions $P(r_a|\mathbf{S}(\lambda, I))$ at $\lambda = 570$ nm and $I=100$ for each type of cone $a \in \{L, M, S\}$. The 3D scatter plot of the samples is shown in Fig.~\ref{fig:3d_scatter_samples}.
\begin{figure}[h]
    \centering
    \includegraphics[width=0.6\textwidth]{figures/Scatter_3dSpace.pdf}
    \caption{3D scatter plot of 1000 samples drawn from the Poisson distributions $P(r_a|\mathbf{S}(\lambda, I))$ at $\lambda = 570$ nm and $I=100$ for each type of cone $a \in \{L, M, S\}$.}
    \label{fig:3d_scatter_samples} 
\end{figure}


\subsection{Another input with different wavelength}

Set another input $\mathbf{S}'(\lambda', I)$ with $\lambda' = \lambda + d\lambda$, where different values of $d\lambda$ are tried to find the approximate value that makes the samples drawn from $P(r_a|\mathbf{S}(\lambda, I))$ and $P(r_a|\mathbf{S}'(\lambda', I))$ have about 30\% overlap and 5\% overlap in the 3D space.

After several trials, it is estimated (by naked eyes) that $d\lambda \approx 8$ nm gives about 30\% overlap and $d\lambda \approx 12$ nm gives about 5\% overlap. As is shown in Fig.~\ref{fig:scatter_overlap}.
\begin{figure}[h]
    \centering
    \includegraphics[width=0.49\textwidth]{figures/Scatter_3dSpace_overlap_30.pdf}
    \includegraphics[width=0.49\textwidth]{figures/Scatter_3dSpace_overlap_5.pdf}
    \caption{3D scatter plot of samples drawn from the Poisson distributions $P(r_a|\mathbf{S}(\lambda, I))$ at $\lambda = 570$ nm (blue) and $P(r_a|\mathbf{S}'(\lambda', I))$ at $\lambda' = \lambda + d\lambda$ (red) for each type of cone $a \in \{L, M, S\}$. Left: $d\lambda \approx 8$ nm for about 30\% overlaps. Right: $d\lambda \approx 12$ nm for about 5\% overlaps.}
    \label{fig:scatter_overlap} 
\end{figure}


\subsection{Decoding the wavelength from a single sample}

Using the samples drawn from $P(r_a|\mathbf{S}(\lambda, I))$ at $\lambda = 570$ nm and $I=100$, the log-likelihood $\log P(r_S, r_M, r_L |  \mathbf{S}=(\hat{\lambda},I))$ is computed for $\hat{\lambda}$ ranging from 400 nm to 700 nm. Here log-likelihood is used instead of likelihood to avoid numerical underflow issues, resulting from multiplying small probabilities.  The value of $\hat{\lambda}$ that maximizes the log-likelihood is taken as the estimated wavelength.

The plot of the log-likelihood, with the estimated $\hat{\lambda}$ that maximizes it, is shown in Fig.~\ref{fig:log_likelihood}. 

\begin{figure}[h]
    \centering
    \includegraphics[width=0.65\textwidth]{figures/log-likelihood.pdf}
    \caption{Log-likelihood $\log P(r_S, r_M, r_L |  \mathbf{S}=(\hat{\lambda},I))$ computed from a single sample drawn from $P(r_a|\mathbf{S}(\lambda, I))$ at $\lambda = 570$ nm and $I=100$, as a function of $\hat{\lambda}$. The red dashed line indicates the true wavelength $\lambda = 570$ nm, and the green dotted line indicates the estimated wavelength $\hat{\lambda}$ that maximizes the log-likelihood.}
    \label{fig:log_likelihood} 
\end{figure}

Then the decoding error for this single sample can be calculated as $\delta\lambda = \hat{\lambda} - \lambda = 572.5-570=2.5$ nm.


\subsection{Histogram of Decoding Errors}

Using multiple samples drawn before, the decoding error $\delta\lambda = \hat{\lambda} - \lambda$ is computed for each sample. The histogram of these decoding errors is shown in Fig.~\ref{fig:histogram_decoding_error}.

\begin{figure}[h]
    \centering
    \includegraphics[width=0.6\textwidth]{figures/histogram_decoding_error.pdf}
    \caption{Histogram of decoding errors $\delta\lambda = \hat{\lambda} - \lambda$ computed from multiple samples drawn from $P(r_a|\mathbf{S}(\lambda, I))$ at $\lambda = 570$ nm and $I=100$.}
    \label{fig:histogram_decoding_error} 
\end{figure}

It can be seen that most of the decoding errors fall in the range of $|\delta \lambda|<10$ nm, so the width of the histogram is about 10 nm, smilar to the $d\lambda = 12$ nm that gives about 5\% overlap in the previous subsection.

\clearpage
\subsection{Decoding errors for a different input wavelength}

Using another input $\mathbf{S}(\lambda, I)$ with $\lambda = 450$ nm and $I=100$, the same procedure is performed to compute the decoding errors $\delta\lambda = \hat{\lambda} - \lambda$ for multiple samples drawn from $P(r_a|\mathbf{S}(\lambda, I))$. The histogram of these decoding errors is shown in Fig.~\ref{fig:histogram_decoding_error_prime}.

\begin{figure}[h]
    \centering
    \includegraphics[width=0.6\textwidth]{figures/histogram_decoding_error_prime.pdf}
    \caption{Histogram of decoding errors $\delta\lambda = \hat{\lambda} - \lambda$ computed from multiple samples drawn from $P(r_a|\mathbf{S}(\lambda, I))$ at $\lambda = 450$ nm and $I=100$.}
    \label{fig:histogram_decoding_error_prime} 
\end{figure}


Also, it is estimated (by naked eyes) that $d\lambda \approx 13$ nm gives about 30\% overlap and $d\lambda \approx 18$ nm gives about 5\% overlap. As is shown in Fig.~\ref{fig:scatter_overlap}.
\begin{figure}[h]
    \centering
    \includegraphics[width=0.49\textwidth]{figures/Scatter_3dSpace_overlap_30_prime.pdf}
    \includegraphics[width=0.49\textwidth]{figures/Scatter_3dSpace_overlap_5_prime.pdf}
    \caption{3D scatter plot of samples drawn from the Poisson distributions $P(r_a|\mathbf{S}(\lambda, I))$ at $\lambda = 450$ nm (blue) and $P(r_a|\mathbf{S}'(\lambda', I))$ at $\lambda' = \lambda + d\lambda$ (red) for each type of cone $a \in \{L, M, S\}$. Left: $d\lambda \approx 13$ nm for about 30\% overlaps. Right: $d\lambda \approx 18$ nm for about 5\% overlaps.}
    \label{fig:scatter_overlap_prime} 
\end{figure}

It can be seen that most of the decoding errors fall in the range of $|\delta \lambda|<19$ nm, so the width of the histogram is about 19 nm, smilar to the $d\lambda = 18$ nm that gives about 5\% overlap in the previous subsection.

\end{document}